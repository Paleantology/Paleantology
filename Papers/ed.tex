\documentclass[]{article}

\begin{document}

\title{Student Research Experiences: Training Biodiversity Scientists Through Data Science}
\author{Wright, Pett, Heath}
\date{Today}
\maketitle

\section{Introduction}

Phylogenetic trees are foundational in the study of evolution. 
We see phylogenetic trees in the study of almost all types of evolutionary studies.
From how we apply medicine to choose an influenza vaccine from year to year, to how we understand the evolution of ecologically-influential traits, phylogeny helps us formulate and test hypotheses about the natural world. \par
Phylogenetic trees can be computed in multiple ways.
Workers using morphological data tend to use parsimony to infer phylogenetic trees.
The parsimony criterion is fairly simplistic: a tree is proposed to explain a data matrix provided by the researcher, and the number of changes implied by the tree is scored.
The tree implying the fewest changes is preferred. 
Similarly, researchers working with large datasets (such as genomic data from microbial populations)  often infer trees using neighbor-joining, in which a matrix of the genetic distances between tips is calculated from the data matrix.
Then, the pairs of tips that have the smallest genetic distances are joined together.
\par
However, most workers in molecular systematics, comparative methods and the study of molecular evolution infer phylogenetic trees using maximum likelihood or Bayesian methods.
We will refer to this framework as statistical phylogenetics.
Statistical phylogenetics makes use of the statistical power of parametric statistics to more robustly infer phylogenetic trees.
While these methods have been consistently supported to perform better at estimating phylogenetic trees, they are also harder to teach.
While parsimony and neighbor-joining are straightfoward counting algorithms, statistical phylogenetics involves understanding the biology of the study organisms and having statistical and computational knowledge. 
For this reason, getting students involved in statistical phylogenetics is a challenge, particularly for early career undergraduates. \par 





\end{document}